\PassOptionsToPackage{subsection=false}{beamerouterthememiniframes} % Omit bar for the subsections
\documentclass[hyperref={colorlinks=true},9pt]{beamer}\usepackage[]{graphicx}\usepackage[]{color}


% character encoding
%\usepackage{alltt}
\usepackage{lmodern}
\usepackage[utf8]{inputenc} 
\usepackage[T1]{fontenc} 
%\usepackage{cmbright} 


% Theme
\usetheme[progressbar=frametitle]{metropolis}
\setbeamercolor{background canvas}{bg=white}

% Progress bar
\setbeamercolor{progress bar}{%
  use=alerted text,
  fg=red,
  bg=alerted text.fg!50!black!30
}

% Make progress bar a bit wider
\makeatletter
\setlength{\metropolis@titleseparator@linewidth}{2pt}
\setlength{\metropolis@progressonsectionpage@linewidth}{2pt}
\setlength{\metropolis@progressinheadfoot@linewidth}{2pt}
\makeatother

% reset page numbers at the end to refine progress in progress bar
\newcommand{\backupbegin}{
   \newcounter{finalframe}
   \setcounter{finalframe}{\value{framenumber}}
}
\newcommand{\backupend}{
   \setcounter{framenumber}{\value{finalframe}}
}



% Theme fonts
\setmainfont{Fira Sans} 
\setsansfont{Fira Sans Light}

% Blocks
\setbeamertemplate{blocks}[rounded][shadow=false]
\setbeamercolor{block title}{bg=black!60,fg=white}
\setbeamercolor{block body}{bg=black!10}

% Footer
%\setbeamertemplate{frame footer}{My custom footer}
\setbeamertemplate{footline}{%
  \parbox{.98\paperwidth}{\vspace*{-8pt}\hspace{5pt}\insertshorttitle~~|~~\insertshortauthor\hfill\insertframenumber}}
%\setbeamertemplate{navigation symbols}{}

\usepackage[ngerman,english]{babel}
\usepackage[babel,german=quotes]{csquotes}
%\setbeamertemplate{headline}{} % hide upper navigation bar
%\usecolortheme{seagull}
\setbeamertemplate{itemize item}[bullet]
\usepackage{graphicx}
\usepackage{color}
\usepackage{multirow}
\usepackage{multicol}
\usepackage{dcolumn}
\usepackage{tabularx}
\usepackage{booktabs}
\usepackage{MnSymbol}
\usepackage{wasysym}
\usepackage[absolute,overlay]{textpos}
\usepackage{listings}
\usepackage{url}
\urlstyle{sf}
\usepackage{hyperref}
\hypersetup{
  pdfauthor={...},
  pdftitle={...},
  pdfsubject={...},
  linkcolor=black,
  urlcolor=darkblue,
  citecolor=darkblue
}
\usepackage{pdfpages}
\usepackage[authoryear,round]{natbib}

%----- Definitions of colors -----%
\definecolor{darkblue}{RGB}{0,0,160}
\definecolor{markfunction}{rgb}{0,0,0.6}
\definecolor{markargument}{RGB}{0, 104, 55}
\definecolor{markcode}{RGB}{26, 26, 26}
\definecolor{markpackage}{RGB}{222,45,38}
\definecolor{markfile}{RGB}{165, 0, 38}
\definecolor{lightblue}{RGB}{158,202,225}
\definecolor{lightgreen}{RGB}{161,217,155}
\definecolor{shadecolor}{rgb}{.97, .97, .97}
\definecolor{messagecolor}{rgb}{0, 0, 0}
\definecolor{warningcolor}{rgb}{1, 0, 1}
\definecolor{errorcolor}{rgb}{1, 0, 0}
\definecolor{darkblue}{RGB}{43,140,190}
\definecolor{darkgreen}{RGB}{43,190,140}
\definecolor{darkred}{rgb}{0.82, 0.1, 0.26}
\definecolor{lightred}{rgb}{1.0, 0.71, 0.76}
\definecolor{darkgreen}{rgb}{0.0, 0.5, 0.0}
\definecolor{lightgreen}{rgb}{0.67, 0.88, 0.69}


%----- Definitions of columntypes -----%
\newcolumntype{N}{>{\scriptsize}c}
\newcolumntype{M}{>{\scriptsize}l}
\newcolumntype{C}[1]{>{\centering\arraybackslash}p{#1}}
\newcolumntype{R}[1]{>{\raggedleft\arraybackslash}p{#1}}
\newcolumntype{L}[1]{>{\raggedright\arraybackslash}p{#1}}
\newcommand{\HRule}{\rule{\linewidth}{0.25mm}}

%----- Definitions of macros in text -----%
%\newcommand{\func}[1]{\texttt{\seqsplit{#1}}}
%\newcommand{\code}[1]{\texttt{\seqsplit{#1}}}
\newcommand{\func}[1]{\textcolor{markfunction}{\footnotesize\texttt{#1}}}
\newcommand{\argu}[1]{\textcolor{markargument}{\footnotesize\texttt{#1}}}
\newcommand{\code}[1]{\textcolor{markcode}{\footnotesize\texttt{#1}}}
\newcommand{\pack}[1]{\textcolor{markpackage}{\textsf{#1}}}
\newcommand{\file}[1]{\textcolor{markfile}{\textit{#1}}}
\newcommand{\R}{\textsf{R}}


%----- Beamer style issues -----%

% Change bullet style
% \useinnertheme{circles}
% \newcommand{\myitem}{\item[\textbullet]}
\beamertemplatenavigationsymbolsempty
\setbeamertemplate{itemize item}{\textbullet}
\def\Tiny{\fontsize{6pt}{6pt}\selectfont}



%----- Code formatting -----%


\lstloadlanguages{R}
%%listing environment for R code input
 \lstdefinestyle{Rinput} {
  %numbers=left,  % where line numbers are displayed
  %stepnumber=1, % spacing between line numbers 
  basicstyle=\ttfamily\scriptsize\color{black},
  backgroundcolor=\color{white},
  aboveskip=0pt, % space above and
  belowskip=0pt, % below listing
  breaklines=true,      % line breaking of long lines.
  breakatwhitespace=false, % allows line breaks only at white space.
  breakindent=0pt,  % no indenting in second line
  breakautoindent=true, % apply intendation
  columns=flexible,    %
  keepspaces=true,
  xleftmargin=0pt, % left indentation
  xrightmargin=0pt, % right indentation
}

%%listing environment for R code output
 \lstdefinestyle{Routput} {
  basicstyle=\ttfamily\scriptsize\color{black},
  commentstyle=\color{black},
  backgroundcolor=\color{white},
  aboveskip=0pt, % space above and
  belowskip=0pt, % below listing
  breaklines=true,      % line breaking of long lines.
  breakatwhitespace=false, % allows line breaks only at white space.
  breakindent=0pt,  % no indenting in second line
  breakautoindent=true, % apply intendation
  columns=flexible,    %
  keepspaces=true,
  xleftmargin=0pt, % left indentation
  xrightmargin=0pt % right indentation
}

%%listing environment for R code output
 \lstdefinestyle{Routputfigure} {
  numbers=left,  % where line numbers are displayed
  numberfirstline=false,
  numberblanklines=false,
  stepnumber=1, % spacing between line numbers
  basicstyle=\ttfamily\scriptsize\color{black},
  commentstyle=\color{black},
  backgroundcolor=\color{white},
  aboveskip=0pt, % space above and
  belowskip=0pt, % below listing
  breaklines=true,      % line breaking of long lines.
  breakatwhitespace=false, % allows line breaks only at white space.
  breakindent=0pt,  % no indenting in second line
  breakautoindent=true, % apply intendation
  columns=flexible,    %
  keepspaces=false,
  xleftmargin=0pt, % left indentation
  xrightmargin=0pt, % right indentation
  frame=single,
  frameround=tttt
}

%%listing environment for HTML code
\lstset{numberstyle=\ttfamily} % define font type of line numbers in listings environment
 \lstdefinestyle{HTML} {
  language=html,
  numbers=left,  % where line numbers are displayed
  numberfirstline=false,
  numberblanklines=false,
  stepnumber=1, % spacing between line numbers
  aboveskip=5pt, % space above and
  belowskip=5pt, % below listing
  extendedchars=true,
  breaklines=true,
  columns=fullflexible,
  showstringspaces=false,
  morestring=[b]",
  morecomment=[s]{<?}{?>},
  morecomment=[s][\color{black}]{<!--}{-->},
  basicstyle=\scriptsize\ttfamily,
  identifierstyle=\color{black}\scriptsize\ttfamily,
  stringstyle=\color{orange}\scriptsize\ttfamily,
  keywordstyle=\color{blue}\scriptsize\bfseries\ttfamily,
  ndkeywordstyle=\color{green}\scriptsize\bfseries\ttfamily,
  commentstyle=\color{brown}\ttfamily,     
  identifierstyle=\color{black},
  keywordstyle=\color{blue}\bfseries,
  frame=single,
  frameround=tttt
}


%%listing environment for XML code
 \lstdefinestyle{XML} {
  language=xml,
  numbers=left,  % where line numbers are displayed
  numberfirstline=false,
  numberblanklines=false,
  stepnumber=1, % spacing between line numbers  
  aboveskip=5pt, % space above and
  belowskip=5pt, % below listing
  extendedchars=true,
  basicstyle=\scriptsize\ttfamily,
  breaklines=true,
  columns=fullflexible,
  showstringspaces=false,
  morestring=[b]",
  morecomment=[s]{<?}{?>},
  morecomment=[s][\color{black}]{<!--}{-->},
  identifierstyle=\color{black},
  stringstyle=\color{black},
  keywordstyle=\bfseries\ttfamily,
  morekeywords={xmlns,version,type,encoding,bond,villain,henchman,id,book,table,author,title,bond_movies,movie,name,year,actors,budget,boxoffice, deadpeople,actor,protagonist,math_wisdom,root,h:head,h:title,t:book,t:author,t:title,xmlns:h,xmlns:t,date,close,volume,open,high,low,company,Apple,document},
  frame=single,
  frameround=tttt
}

%%listing environment for DTD code
 \lstdefinestyle{DTD} {
  numbers=left,  % where line numbers are displayed
  numberfirstline=false,
  numberblanklines=false,
  stepnumber=1, % spacing between line numbers 
  aboveskip=5pt, % space above and
  belowskip=5pt, % below listing
  extendedchars=true,
  basicstyle=\scriptsize\ttfamily,
  breaklines=true,
  columns=fullflexible,
  showstringspaces=false,
  morestring=[b]",
  morecomment=[s]{<?}{?>},
  morecomment=[s][\color{black}]{<!--}{-->},
  identifierstyle=\color{black},
  stringstyle=\color{black},
  frame=single,
  frameround=tttt
}

%%listing environment for JSON code
 \lstdefinestyle{JSON} {
  numbers=left,  % where line numbers are displayed
  numberfirstline=false,
  numberblanklines=false,
  stepnumber=1, % spacing between line numbers 
  aboveskip=5pt, % space above and
  belowskip=5pt, % below listing
  extendedchars=true,
  basicstyle=\scriptsize\ttfamily,
  breaklines=true,
  columns=fullflexible,
  showstringspaces=false,
  morestring=[b]",
  morecomment=[s]{<?}{?>},
  morecomment=[s][\color{black}]{<!--}{-->},
  identifierstyle=\color{black},
  stringstyle=\color{black},
  frame=single,
  frameround=tttt
}

%%listing environment for SVG code
 \lstdefinestyle{SVG} {
  numbers=left,  % where line numbers are displayed
  numberfirstline=false,
  numberblanklines=false,
  stepnumber=1, % spacing between line numbers 
  aboveskip=5pt, % space above and
  belowskip=5pt, % below listing
  extendedchars=true,
  basicstyle=\scriptsize\ttfamily,
  breaklines=true,
  columns=fullflexible,
  showstringspaces=false,
  morestring=[b]",
  morecomment=[s]{<?}{?>},
  morecomment=[s][\color{black}]{<!--}{-->},
  identifierstyle=\color{black},
  stringstyle=\color{black},
  keywordstyle=\bfseries\ttfamily,
  morekeywords={svg,ellipse,text,circle,rect,path},
  frame=single,
  frameround=tttt
}

%%listing environment for RSS code
 \lstdefinestyle{RSS} {
  numbers=left,  % where line numbers are displayed
  numberfirstline=false,
  numberblanklines=false,
  stepnumber=1, % spacing between line numbers 
  aboveskip=5pt, % space above and
  belowskip=5pt, % below listing
  extendedchars=true,
  basicstyle=\scriptsize\ttfamily,
  breaklines=true,
  columns=fullflexible,
  showstringspaces=false,
  morestring=[b]",
  morecomment=[s]{<?}{?>},
  morecomment=[s][\color{black}]{<!--}{-->},
  identifierstyle=\color{black},
  stringstyle=\color{black},
  keywordstyle=\bfseries\ttfamily,
  morekeywords={rss,channel,description,title,item,link,author,category,enclosure,guid,image,language,pubDate,source,ttl,lastBuildDate,yweather},
  frame=single,
  frameround=tttt
}

%%listing environment for HTTP code
 \lstdefinestyle{HTTP} {
  numbers=left,  % where line numbers are displayed
  numberfirstline=false,
  numberblanklines=false,
  stepnumber=1, % spacing between line numbers 
  aboveskip=5pt, % space above and
  belowskip=5pt, % below listing
  extendedchars=true,
  basicstyle=\scriptsize\ttfamily,
  breaklines=true,
  columns=fullflexible,
  showstringspaces=false,
  morestring=[b]",
  morecomment=[s]{<?}{?>},
  morecomment=[s][\color{black}]{<!--}{-->},
  identifierstyle=\color{black},
  stringstyle=\color{black},
  keywordstyle=\bfseries\ttfamily,
  morekeywords={GET,POST,HEAD,PUT,DELETE,TRACE,OPTIONS,CONNECT},
  frame=single,
  frameround=tttt
}

%%listing environment for SQL code
 \lstdefinestyle{SQL} {
  numbers=left,  % where line numbers are displayed
  numberfirstline=false,
  numberblanklines=false,
  stepnumber=1, % spacing between line numbers 
  language=sql,
  aboveskip=5pt, % space above and
  belowskip=5pt, % below listing
  extendedchars=true,
  basicstyle=\scriptsize\ttfamily,
  breaklines=true,
  columns=fullflexible,
  keepspaces=true,
  showstringspaces=false,
  identifierstyle=\color{black},
  stringstyle=\color{black},
  keywordstyle=\bfseries\ttfamily,
  morekeywords={IDENTIFIED,TO,WITH,AUTO_INCREMENT,DATETIME,DATE,BINARY,REFERENCES,MODIFY},
  frame=single,
  frameround=tttt
}

%%listing environment for BASH code
 \lstdefinestyle{Bash} {
  numbers=left,  % where line numbers are displayed
  numberfirstline=false,
  numberblanklines=false,
  stepnumber=1, % spacing between line numbers 
  language=java,
  aboveskip=5pt, % space above and
  belowskip=5pt, % below listing
  extendedchars=true,
  basicstyle=\scriptsize\ttfamily,
  breaklines=true,
  columns=fullflexible,
  keepspaces=true,
  showstringspaces=false,
  identifierstyle=\color{black},
  stringstyle=\color{black},
  keywordstyle=\bfseries\ttfamily,
  frame=single,
  frameround=tttt
}

%%listing environment for simple TXT files
 \lstdefinestyle{TXT} {
  numbers=left,  % where line numbers are displayed
  numberfirstline=false,
  numberblanklines=false,
  stepnumber=1, % spacing between line numbers  
  aboveskip=5pt, % space above and
  belowskip=5pt, % below listing
  extendedchars=true,
  basicstyle=\scriptsize\ttfamily,
  breaklines=true,
  columns=fullflexible,
  keepspaces=true,
  showstringspaces=false,
  identifierstyle=\color{black},
  stringstyle=\color{black},
  keywordstyle=\bfseries\ttfamily,
  frame=single,
  frameround=tttt
} 

%%listing environment for JavaScript code
 \lstdefinestyle{JavaScript} {
  numbers=left,  % where line numbers are displayed
  numberfirstline=false,
  numberblanklines=false,
  stepnumber=1, % spacing between line numbers  
  aboveskip=5pt, % space above and
  belowskip=5pt, % below listing
  extendedchars=true,
  basicstyle=\scriptsize\ttfamily,
  breaklines=true,
  columns=fullflexible,
  showstringspaces=false,
  morestring=[b]',
  morestring=[b]",
  sensitive=false,
  comment=[l]{//},
  morecomment=[s]{/*}{*/},
  identifierstyle=\color{black},
  stringstyle=\color{black},
  keywordstyle=\bfseries\ttfamily,
  morekeywords={typeof, new, true, false, catch, function, return, null, catch, switch, var, if, in, while, do, else, case, break},
  frame=single,
  frameround=tttt
}


%% maxwidth is the original width if it is less than linewidth
%% otherwise use linewidth (to make sure the graphics do not exceed the margin)
\makeatletter
\def\maxwidth{ %
  \ifdim\Gin@nat@width>\linewidth
    \linewidth
  \else
    \Gin@nat@width
  \fi
}
\makeatother

\definecolor{fgcolor}{rgb}{0.345, 0.345, 0.345}
\newcommand{\hlnum}[1]{\textcolor[rgb]{0.686,0.059,0.569}{#1}}%
\newcommand{\hlstr}[1]{\textcolor[rgb]{0.192,0.494,0.8}{#1}}%
\newcommand{\hlcom}[1]{\textcolor[rgb]{0.678,0.584,0.686}{\textit{#1}}}%
\newcommand{\hlopt}[1]{\textcolor[rgb]{0,0,0}{#1}}%
\newcommand{\hlstd}[1]{\textcolor[rgb]{0.345,0.345,0.345}{#1}}%
\newcommand{\hlkwa}[1]{\textcolor[rgb]{0.161,0.373,0.58}{\textbf{#1}}}%
\newcommand{\hlkwb}[1]{\textcolor[rgb]{0.69,0.353,0.396}{#1}}%
\newcommand{\hlkwc}[1]{\textcolor[rgb]{0.333,0.667,0.333}{#1}}%
\newcommand{\hlkwd}[1]{\textcolor[rgb]{0.737,0.353,0.396}{\textbf{#1}}}%

\usepackage{framed}
\makeatletter
\newenvironment{kframe}{%
 \def\at@end@of@kframe{}%
 \ifinner\ifhmode%
  \def\at@end@of@kframe{\end{minipage}}%
  \begin{minipage}{\columnwidth}%
 \fi\fi%
 \def\FrameCommand##1{\hskip\@totalleftmargin \hskip-\fboxsep
 \colorbox{shadecolor}{##1}\hskip-\fboxsep
     % There is no \\@totalrightmargin, so:
     \hskip-\linewidth \hskip-\@totalleftmargin \hskip\columnwidth}%
 \MakeFramed {\advance\hsize-\width
   \@totalleftmargin\z@ \linewidth\hsize
   \@setminipage}}%
 {\par\unskip\endMakeFramed%
 \at@end@of@kframe}
\makeatother


\newenvironment{knitrout}{}{} % an empty environment to be redefined in TeX

\ifdefined\knitrout
  \renewenvironment{knitrout}{\begin{scriptsize}\setlength{\topsep}{0mm}}{\end{scriptsize}}
\else
\fi
